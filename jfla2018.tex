% Conference: JFLA 2018, short paper
%
% Page limit: 6 pages, including references

\documentclass[a4paper]{easychair}

\usepackage[utf8]{inputenc}
\usepackage[T1]{fontenc}
\usepackage[francais]{babel}
\usepackage{mathpartir}
\usepackage{wasysym}
\usepackage{listings}
\usepackage{alltt}

% % for review: page and line numbers "if possible"
% \usepackage{lineno}
% \linenumbers
% \pagestyle{plain}

\title{Génération aléatoire de programmes guidée par la vivacité}
% \title{Liveness-Driven Random Program Generation}
\author{Gergö Barany\inst{1} et Gabriel Scherer\inst{2}}
\institute{
Inria Paris \\
\email{gergo.barany@inria.fr}
\and
Inria Saclay \\
\email{gabriel.scherer@gmail.com}
}

\titlerunning{Génération aléatoire guidée par la vivacité}
\authorrunning{G. Barany et G. Scherer}

% macros go here
\newcommand\ldrgen{\texttt{ldrgen}}
\newcommand{\liveout}[1]{\ensuremath{#1^{\circ}}}
\newcommand{\livein}[1]{\ensuremath{#1^{\bullet}}}
\newcommand{\triple}[3]{\ensuremath{
    \left\langle#1\right\rangle \ #2 \ \left\langle#3\right\rangle
}}
\def\union{\cup}

% en Français la norme est de mettre en italique
% les expressions en langue étragère
\newcommand{\anglais}[1]{\textit{#1}}

% make listings not super-ugly
\lstset{language=[Objective]Caml,columns=fullflexible,emphstyle=\emph,
        mathescape=true,showstringspaces=false,
        stringstyle=\texttt,commentstyle=\color{violet}}
% Gabriel: j'ai mis "violet" parce que:
% - utiliser de l'italique (mode par défaut) rend l'utilisation du
% mode maths dans les commentaires peu lisible
% - j'aurais préféré un gris foncé mais je n'ai pas trouvé comment le
% nommer sans changer les options des paquets LaTeX, donc j'ai été
% lâche; 'violet' rend plutôt bien en fait.

% WTF, listings?
% https://stackoverflow.com/questions/1116266/listings-in-latex-with-utf-8-or-at-least-german-umlauts
\lstset{%
literate=%
{é}{{\'{e}}}1
{è}{{\`{e}}}1
{ê}{{\^{e}}}1
{ë}{{\¨{e}}}1
{û}{{\^{u}}}1
{ù}{{\`{u}}}1
{â}{{\^{a}}}1
{à}{{\`{a} }}1  % hack super débile
{î}{{\^{i}}}1
{ô}{{\^{o}}}1
{ç}{{\c{c}}}1
{Ç}{{\c{C}}}1
{É}{{\'{E}}}1
{Ê}{{\^{E}}}1
{À}{{\`{A}}}1
{Â}{{\^{A}}}1
{Î}{{\^{I}}}1
}

\begin{document}
\maketitle
\thispagestyle{plain}

\begin{abstract}
Les programmes générés aléatoirement sont un bon moyen de tester des
compilateurs et des outils d'analyse de logiciel. Des centaines de bogues
ont été trouvés dans des compilateurs C très utilisés (GCC, Clang) par des
tests aléatoires. Pourtant, les générateurs existants peuvent générer
beaucoup de code mort (dont les résultats ne sont jamais utilisés). Compiler
un tel programme laisse relativement peu de possibilités d'exercer les
optimisations complexes du compilateur.

Pour résoudre ce problème, nous proposons la génération aléatoire de
programmes guidée par la vivacité. Dans cette approche, le programme aléatoire
est construit \anglais{bottom-up}, en combinaison avec une analyse de flot de
données structurelle pour assurer que le système ne génère jamais de code
mort.

L'algorithme est implanté dans un greffon pour l'outil Frama-C. Nous
l'évaluons en comparaison avec Csmith, le générateur aléatoire standard pour
le langage C. Les programmes générés par notre outil compilent vers une plus
grande quantité de code machine, avec une plus grande variété
d'instructions.

Ce papier est une version courte d'un article présenté à LOPSTR~2017.
\end{abstract}

% \keywords{code generation, random testing, data-flow analysis, program
% optimization}


\section{Motivation}

Les compilateurs pour les langages de programmation modernes sont
compliqués et difficiles à comprendre. Malgré les avances en vérification de
compilateurs~\cite{leroy-2009,tan-2016}, la plupart des compilateurs
réalistes ne sont pas vérifiés formellement. Il faut des tests pour acquérir une confiance en leur correction partielle. Une approche populaire est le test aléatoire,
fait avec des fichiers d'entrée produits par un générateur aléatoire de
programmes. L'outil le plus connu dans ce domaine est Csmith, un générateur
aléatoire de programmes C. Csmith est très puissant~: il a trouvé des
centaines de bogues dans des compilateurs très utilisés comme GCC et Clang,
et même quelques bogues dans des parties non vérifiées du compilateur vérifié
CompCert~\cite{csmith-2011}.

Le travail présenté ici provient d'un projet de test aléatoire des optimisations effectuées par des compilateurs C.
Nous avions besoin d'un générateur aléatoire produisant de grandes fonctions effectuant des calculs compliqués, sans appels
de fonctions, pour pouvoir analyser le code machine optimisé généré par des
compilateurs différents. Csmith peut être configuré pour cet usage, mais
nous avons découvert que le compilateur enlève presque tous les calculs
présents dans le code source généré par Csmith. Même des boucles imbriquées entières
disparaissent souvent dans le code machine compilé à partir du code source
généré par Csmith. Ceci nous empêche de faire des analyses intéressantes sur
le code machine final.

Le problème vient du fait que Csmith génère des programmes qui ne contiennent
presque que du~\emph{code mort}~: des calculs dont les résultants ne sont
jamais utilisés. Les compilateurs enlèvent tout le code mort, laissant très
peu de possibilités d'appliquer nos analyses.

Cet article décrit \ldrgen, un nouveau générateur de code C guidée par une
analyse de vivacité pour éviter de générer du code mort.


\section{Analyse de la vivacité}

Une variable est~\emph{vivante} à une position donné d'un programme si sa valeur a
cette position peut être utilisée plus tard.
% Gabriel: par "point", est-ce qu'on parle d'une position (statique) dans le code,
% ou d'une étape de l'exécution (dynamique) du programme ? Dans le premier cas,
% que veut dire "plus tard" ?
Sinon, elle est~\emph{morte}. De la
même façon, une affectation~\(v = e\) est vivante si elle définit une
variable vivante, et morte sinon. Par exemple, dans un morceau de code
comme~\texttt{x = y + z; return y}, la variable~\texttt{x} est morte juste
après son affectation car sa valeur n'est jamais utilisée. L'affectation
elle-même est donc morte aussi, et elle peut être enlevée du programme sans
en changer le sens.

Notre objectif est de ne générer aucun code mort, et en particulier, de ne
générer que des programmes dans lesquels toutes les affectations sont
vivantes.

\subsection{Analyse itérative}

L'analyse de la vivacité est un analyse classique de flot de
données~\cite{nielson.etal-1999}. Elle parcourt le graphe de flot de
contrôle du programme en arrière et itère jusqu'à trouver un point fixe.
Chaque instruction~\(S\) est liée à deux ensembles de variables vivantes :
l'ensemble~\(\livein{S}\) de variables vivantes juste avant~\(S\), et
l'ensemble~\(\liveout{S}\) de celles vivantes juste après~\(S\).
Ces ensembles sont liés par des fonctions~\(f_S\) spécifiques à
l'instruction~\(S\) qui décrivent la propagation de l'information. Pour une
affectation~\(v = e\) de la valeur d'une expression~\(e\) contenant
l'ensemble de variables~\(FV(e)\) à la variable~\(v\), la propagation est
faite par l'équation suivante:
\[
\livein{S} = f_{v = e}= (\liveout{S} \setminus \{v\}) \union FV(e)
\]

Pour une instruction de branchement avec une condition~\(c\) (comme
\(S = \mbox{\texttt if (\(c\)) \dots}\)), l'analyse unit l'information de
tous les successeurs dans le graphe de flot de contrôle et rajoute les
variables de la condition:
\[
\livein{S} = \bigcup_{S_i \in \mathit{succ}(S)} \livein{S_i} \cup FV(c)
\]

Les boucles génèrent des systèmes d'équations récursifs. Ces systèmes sont
traditionnellement résolus par itération jusqu'à obtenir un point fixe
minimal comme solution.

\subsection{Analyse structurelle de programmes sans code mort}

L'analyse de flot de données sur le graphe de flot de contrôle est pratique
pour identifier le code mort dans un programme donné. Cependant, notre
objectif n'est pas d'analyser des programmes donnés mais d'utiliser
l'analyse pour guider la génération de code afin d'éviter de générer du code
mort. Nous voudrions éviter de construire un graphe de flot de contrôle et
de calculer les points fixes pour les boucles par itération.

Nous avons donc décidé d'essayer une approche structurelle basée sur la
structure de l'arbre de syntaxe abstraite. Les approches structurelles ne
sont appropriées qu'aux programmes structurés (sans~\texttt{goto}
etc.), mais cela suffit pour notre générateur.

L'analyse est présentée dans la figure~\ref{fig:rules} comme un système de
règles d'inférence. L'information inférée sont des
triples~\(\triple{\livein{S}}{S}{\liveout{S}}\) d'une instruction~\(S\) et
ses ensembles de variables vivantes avant et après l'instruction
(\(\livein{S}\), \(\liveout{S}\)). Les règles contiennent deux sortes
d'hypothèses~: L'une sont les équations de propagation de l'information de
l'analyse de flot de données. En particulier, la règle~\textsc{Assign} pour
les affectations de la forme~\(v = e\) a comme hypothèse
l'équation~\(\livein{S} = (\liveout{S} \setminus \{v\}) \union FV(e)\),
comme dans le cas de l'analyse de flot de données itérative.

La deuxième sorte d'hypothèses sont les conditions pour assurer que les
programmes qui contiennent du code mort ne sont pas acceptés par le système.
La condition la plus importante est celle de la règle~\textsc{Assign}: Une
affectation d'une variable~\(v\) ne peut être acceptée que si~\(v\) est
vivante après l'affectation. De la même façon, la condition~\(S_1 \neq \{\}
\lor S_2 \neq \{\}\) pour~\textsc{If} assure que les instructions de
branchement inutiles comme~\texttt{if (x) \{\} else \{\}} ne sont pas
acceptées.

Un programme~\(S\) ne contient pas de code mort si le
triple~\(\triple{\livein{S}}{S}{\emptyset}\) peut être dérivée dans le
système de règles d'inférence.

Le traitement de boucles est plus compliqué à cause de la dépendance
cyclique entre l'ensemble~\(\livein{B}\) de variables vivantes au début du
corps~\(B\) de la boucle, et l'ensemble~\(\liveout{B}\) de variables
vivantes à la fin du corps de la boucle. La condition de minimalité exprime
que la solution désirée est un point fixe minimale du système d'équations.
Un point fixe unique minimale existe toujours~\cite{nielson.etal-1999}.

\begin{figure}
\[
\inferrule*[left=Return]{
}{
    \triple{\{v\}}{\mathtt{return}\ v}{\emptyset}
}
\qquad
\inferrule*[left=Skip]{
    L \neq \emptyset
}{
    \triple{L}{\mathtt{\{\}}}{L}
}
\]

\[
\inferrule*[left=Assign]{
    v \in \liveout{S} \\
    \livein{S} = (\liveout{S} \setminus \{v\}) \union FV(e)
}{
    \triple{\livein{S}}{v\ \mathtt{=}\ e}{\liveout{S}}
}
\]

\[
\inferrule*[left=Sequence]{
    \triple{\livein{S_1}}{S_1}{\livein{S_2}} \\
    \triple{\livein{S_2}}{S_2}{\liveout{S_2}} \\
    \livein{S_2} \neq \emptyset
}{
    \triple{\livein{S_1}}{S_1\ \mathtt{;}\ S_2}{\liveout{S_2}}
}
\]

\[
\inferrule*[left=If]{
    \triple{\livein{S_1}}{S_1}{\liveout{S}} \\
    \triple{\livein{S_2}}{S_2}{\liveout{S}} \\
    \livein{S} = \livein{S_1} \union \livein{S_2} \union FV(c) \\
    S_1 \neq \{\} \lor S_2 \neq \{\}
}{
    \triple{\livein{S}}
           {\mathtt{if\ (}c\mathtt{)}\ S_1\ \mathtt{else}\ S_2}
           {\liveout{S}}
}
\]

\[
\inferrule[While]{
    \triple{\livein{B}}{B}{\liveout{B}} \\
    \liveout{B} = \liveout{S} \union \livein{B} \union FV(c)
        \mbox{ (minimal)} \\
    \livein{S} = \livein{B} \union \liveout{S} \\
    \liveout{S} \neq \emptyset
}{
    \triple{\livein{S}}
           {\mathtt{while\ (}c\mathtt{)}\ B}
           {\liveout{S}}
}
\]
\caption{Système de règles d'inférence pour la recognition de programmes
sans code mort}
\label{fig:rules}
\end{figure}


\section{Génération de code non mort}

Les règles d'inférence peuvent être interprétées comme un générateur
exécutable aléatoire (ou exhaustif) de programmes sans code mort. Comme
l'analyse traditionnelle de vivacité, la génération se fait en arrière,
c'est-à-dire, dans la direction opposante au flot de contrôle.

Le générateur commence par générer une variable~\(v\) et une
instruction~\texttt{return \(v\)}. L'ensemble de variables vivantes avant
cette instruction est~\(L = \{v\}\). Le générateur applique des fonctions
pour générer des nouvelles instructions aléatoires étant donné un ensemble
de variables vivantes. Les nouvelles instructions~\(S\) sont préfixées au
programme, et l'ensemble~\(L\) est mis à jour selon la fonction
correspondante~\(f_S\). Cet ensemble guide le générateur~: En particulier,
pour générer une affectation d'une variable~\(v\), \(L\) doit contenir~\(v\)
à ce point dans le programme. L'itération se termine après un nombre
prédéfini d'instructions, ou si jamais l'ensemble~\(L\) devient vide. La
figure~\ref{fig:generator} contient le pseudo-code du générateur dans un
langage fonctionnel.

\begin{figure}
\lstset{emph={v,L,S,e,code,c,t,f,V,B}}
\begin{lstlisting}
let random_statements L code =
    if L $= \emptyset$ then (code, L)
    else
        let (S, $L'$) = random_statement L in
        random_statements $L'$ (S :: code)

let random_statement L =
    let statement_generator = random_select [assignment; branch; loop] in
    statement_generator L

let assignment L =
    let v = random_select L in
    let e = random_expression () in
    ("$v$ = $e$", $(L \setminus \{v\}) \union FV(e)$)

let branch L =
    let (t, $L_1$) = random_statements L [] in
    let (f, $L_2$) = random_statements L [] in
    let c = random_expression () in
    ("if ($c$) $t$ else $f$", $L_1 \union L_2 \union FV(c)$)

let loop L =
    (* Générer un point fixe pour la boucle, puis la boucle elle-même. *)
    let c = random_expression () in
    let $B'$ = random_variable_set () in
    let (code, L') = random_statements $(L \union B' \union FV(c))$ [] in
    let $V = \{ b \in B' \mid b \notin L' \}$ in (* les $b$ pas dans $code$ *)
    if $V = \emptyset$ then
        ("while ($c$) $code$", $L' \union L$)
    else
        let e = random_expression_on_variables V in
        let v = random_select $L'$ in
        let code' = "$v$ = $e$" :: code in
        ("while ($c$) $code'$", $(L' \setminus \{v\}) \union V \union L$)

(* Commencer la génération à la fin du programme. *)
let v = random_variable ()
let (code, L) = random_statements {v} ["return $v$"]
\end{lstlisting}
\caption{Pseudo-code du générateur aléatoire guidée par la vivacité.}
\label{fig:generator}
\end{figure}

Lors de la génération des boucles, nous voudrions éviter la construction
d'un graphe de flot de contrôle et l'itération jusqu'à un point fixe de la
boucle. En effet, nous ne pouvons pas facilement faire une analyse itérative
d'un corps de boucle qui n'existe pas encore.

Pour éviter ce problème cyclique, notre générateur choisit d'abord un
ensemble de variables qui doivent devenir vivantes au début du corps de la
boucle, puis génère le corps lui-même. Finalement, il peut générer des
instructions supplémentaires pour assurer la validité du choix de variables.

Plus précisément, nous cherchons à générer une liste~\(B\) d'instructions et
un ensemble~\(\livein{B}\) tels que~\(\livein{B} = f_B(\liveout{B})\)
et~\(\liveout{B} = \liveout{S} \union \livein{B} \union FV(c)\), étant
donnés~\(\liveout{S}\) et une expression~\(c\) pour la condition de la
boucle. Nous générons un ensemble~\(B'\) de nouvelles variables pour
représenter les variables intéressantes pour le cas des boucles~: Les
variables utilisées dans la boucle avec une valeur qui peut venir d'une
itération précédente de la même boucle. Dit autrement, ces variables peuvent
être affectées par le corps de la boucle, mais elles doivent aussi avoir
d'utilisations dans le corps de la boucle.

Après avoir choisi l'ensemble~\(B'\), nous pouvons générer une liste
d'instructions sous l'hypothèse que les variables dans
l'ensemble~\(\liveout{S} \union \livein{B} \union FV(c)\) sont vivantes à sa
fin. Le générateur rend cette liste d'instructions (appelée~\(code\) en
figure~\ref{fig:generator}) et l'ensemble~\(L'\) de variables vivantes à son
début. Pour que~\(code\) puisse servir comme corps de boucle, il suffit
d'assurer que notre choix de~\(B'\) satisfait la condition ci-dessus~:
Chaque variable~\(b \in B'\) doit être utilisée, et elle doit être vivante
au début de la boucle. Il suffit de préfixer~\(code\) avec une instruction
qui utilise toutes les variables~\(b\) où ce n'est pas encore le cas pour
obtenir une liste finale~\(B\) et un ensemble~\(\livein{B}\) qui satisfont
\(\livein{B} = f_B(\liveout{B})\), \(\liveout{B} = \liveout{S} \union B'
\union FV(c)\), et~\(B' \subseteq \livein{B}\).


\section{Implantation}

Notre générateur \ldrgen\ est implantée dans un greffon pour Frama-C, ce qui
est un outil extensible pour l'analyse et transformation de logiciels écrits
en C~\cite{frama-c}. \ldrgen\ est un logiciel libre, disponible
à~\url{https://github.com/gergo-/ldrgen}. L'implantation du générateur
comprend environ~600 lignes d'OCaml, suivant la structure du pseudo-code en
figure~\ref{fig:generator}. Pour l'instant, l'outil génère un sous-ensemble
du langage~C avec des opérations arithmétiques et bit à bit sur tous les
types de base, ainsi que des branchements avec~\texttt{if}, des
boucles~\texttt{while}, et des boucles~\texttt{for} d'une forme restreinte
pour calculer une opération de réduction sur des tableaux de taille fixe. Il
ne génère pas encore de~\texttt{struct} ni de l'arithmétique sur des
pointeurs.

% boucles for, configuration, flot de contrôle...

Une description plus complète se trouve dans la version originale de cet
article~\cite{ldrgen-2017}.


\section{Évaluation}

Nous évaluons le code généré par \ldrgen\ en comparaison avec~Csmith. Pour
ces expériences, nous avons configuré Csmith pour générer une seule
fonction sans appels à autres fonctions. (Le mode standard de Csmith est de
générer une application complète et de cacher la plupart des opérations
arithmétiques dans des fonctions auxiliaires qui protègent des
débordements.)

Nos résultats pour le code généré par~1000 appels de chaque outil sont
résumés dans la table~\ref{tab:results}. La première partie de la table
montre que nous avons choisi des options de configuration pour les deux
outils pour générer des quantités comparables de code source. Ceci nous
permet de faire une comparaison juste des programmes générés.

La deuxième partie de la table montre le nombre d'instructions machine
émises par GCC (\texttt{-O3}, sur x86-64) pour le code source généré par
chaque outil. La médiane de 15 instructions machine pour Csmith nous
démontre qu'au moins la moitié des fonctions générés par Csmith ont une
taille triviale. Ces fonctions très petites sont exactement ce que nous
voulions éviter car elles ne sont pas intéressantes pour notre analyse de
code machine. En moyenne, le code généré par Csmith compile vers environ une
seule instruction machine par~10 lignes de code source. En revanche, avec
\ldrgen\ nous obtenons environ~2,5 instructions machine par ligne de code.

La dernière partie de la table concerne la variété d'instructions générées.
Nous regardons les nombres d'instructions machine différentes par fonction
générée et au total. Même la fonction la plus diverse générée par Csmith
contient moins de variété que la fonction médiane générée par \ldrgen. Au
total, le code généré par \ldrgen\ permet une couverture du jeu
d'instructions~40\,\% plus élevé par rapport au code généré par Csmith. Par
inspection des ensembles d'instructions uniques, nous avons trouvé que
presque toute la différence vient des instructions vectorielles (SIMD) qui
sont émises pour les boucles~\texttt{for} sur des tableaux générées par
\ldrgen. Csmith sait générer des boucles similaires, mais leurs résultats ne
sont presque jamais utilisés.

\begin{table}
\caption{Comparaison du code généré par Csmith et \ldrgen\ en 1000 appels de
chacun.}
\label{tab:results}

\vskip 1em
\centering
\begin{tabular}{l@{\ } | @{\ }l @{\quad} r @{\quad} r @{\quad} r @{\quad} r}

& générateur & min & médiane & max & total \\
\hline
lignes de code & Csmith
    & 25 & 368.5 & 2953 & 459021 \\
& \ldrgen
    & 12 & 411.5 & 1003 & 389939
    \\[1ex]
instructions & Csmith
    & 1 &  15.0 & 1006 &   45606 \\
& \ldrgen
    & 1 & 952.5 & 4420 & 1063503
    \\[1ex]
opcodes uniques & Csmith
    & 1 &  8 &  74 & 146 \\
& \ldrgen
    & 1 & 95 & 124 & 204
    \\
\end{tabular}
\end{table}


\section{Conclusions}

Nous avons présenté \ldrgen, un générateur de code source C aléatoire. Le
but de ce générateur est de ne jamais générer du code mort, afin d'obtenir
des grandes quantités d'instructions machine variées générés par des
compilateurs optimisants.
%
\ldrgen\ est guidé par une analyse structurelle de vivacité lors de la
génération de code. Une nouvelle approche à l'analyse de flot de données
structurelle interprétée comme système d'inférence nous permet de faire
l'analyse sans devoir implémenter l'itération jusqu'à un point fixe.
%
Par rapport au générateur de code Csmith, \ldrgen\ génère une plus
grande quantité de code machine plus varié.


\bibliography{lopstr2017}
\bibliographystyle{alpha}

\end{document}
